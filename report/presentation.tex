\documentclass[UTF8]{ctexbeamer}
\usepackage{physics}
\usepackage{amsmath, amssymb, mathtools}
\usepackage{tikz}
\usepackage{mathdots}
\usepackage{yhmath}
\usepackage{cancel}
\usepackage{color}
\usepackage{siunitx}
\usepackage{array}
\usepackage{multirow}
\usepackage{amssymb}
\usepackage{textcomp, gensymb}
\usepackage{tabularx}
\usepackage{extarrows}
\usepackage{booktabs}
\usetikzlibrary{fadings}
\usetikzlibrary{patterns}
\usetikzlibrary{shadows.blur}
\usetikzlibrary{shapes}
\usepackage{listings}
\usepackage{hyperref}

%Information to be included in the title page:
\title{二维正方晶格伊辛模型的蒙特卡洛模拟}
\author{吴晋渊 18307110155}
\institute{复旦大学物理学系}
\date{2021}

\usetheme{Madrid}

\newcommand{\concept}[1]{\textbf{#1}}

\begin{document}

\frame{\titlepage}

\begin{frame}
\frametitle{Finite size scaling and data collapsing}

有限大小的系统中可观测量的行为

\begin{itemize}
    \item 独立参数:体系边长$L$和归一化温度$t = (T - T_\text{c}) / T_\text{c}$;等价地说,$L$和($L \to \infty$时的)关联长度$\xi \propto \abs*{t}^{- \nu}$
    \item $L$足够长,系统足够接近临界点:正比于“系统中各个自由度求和”的物理量$Q \sim \int \dd[d]{\vb*{r}} \phi^n (\grad{\phi})^m$也具有关于$L$的标度不变性:
    \[
        Q(t, L) = L^\sigma f(\xi / L) = L^{\sigma} f(\abs*{t}^{- \nu} / L) = L^\sigma g(\underbrace{t L^{1 / \nu}}_{= (\xi / L)^{- 1 / \nu} \coloneqq x}).
    \]
    \item $L \to \infty$时$Q \sim \abs*{t}^{- \kappa}$,于是只能$g(t L^{1 / \nu}) \sim (t L^{1 / \nu})^{- \kappa}$;由于$Q$此时没有依赖$L$,只能是$\sigma = \kappa / \nu$。于是
    \begin{equation}
        Q(t, L) = L^{\kappa / \nu} g(t L^{1 / \nu}).
    \end{equation}
\end{itemize}    

\end{frame}

\begin{frame}
\frametitle{Finite size scaling and data collapsing}

有限大小的系统的伪临界点

\begin{itemize}
    \item $L$有限时$Q$在$t=0$时并不出现真正的不连续性
    \item 单次蒙卡模拟时固定$L$扫描$t$:$Q(t, L)$在
    \begin{equation}
        t = t_\text{max}, \quad t_\text{max} L^{1 / \nu} = x_\text{max}
    \end{equation}
    时最大(或者取零,例如$Q$为磁化强度,不发散时),其中$x_\text{max}$让$g(x)$最大。
    \item $L$固定,$g(x)$最大则$Q(t, L)$最大,为
    \begin{equation}
        Q_\text{max}(L) \sim L^{\kappa / \nu}, \quad t_\text{max} L^{1 / \nu} \sim 1 \Rightarrow t_\text{max} \sim L^{- 1 / \nu}.
    \end{equation}
    \item 因此可以将固定$L$时$Q(t = (T - T_\text{c}) / T_\text{c}, L)$取最大值的$T$当成“伪临界点”,其上的scaling关系是关于$L$而不是关于$\xi$的。拟合
    \begin{equation}
        T = T_\text{c} + \alpha L^{- 1 / \nu}
    \end{equation}
    可以得到$T_\text{c}$和$\nu$两个参数。
\end{itemize}

\end{frame}

\begin{frame}
\frametitle{实现}

    

\end{frame}

\end{document}