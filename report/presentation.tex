\documentclass[UTF8]{ctexbeamer}
\usepackage{physics}
\usepackage{amsmath}
\usepackage{tikz}
\usepackage{mathdots}
\usepackage{yhmath}
\usepackage{cancel}
\usepackage{color}
\usepackage{siunitx}
\usepackage{array}
\usepackage{multirow}
\usepackage{amssymb}
\usepackage{textcomp, gensymb}
\usepackage{tabularx}
\usepackage{extarrows}
\usepackage{booktabs}
\usetikzlibrary{fadings}
\usetikzlibrary{patterns}
\usetikzlibrary{shadows.blur}
\usetikzlibrary{shapes}
\usepackage{listings}
\usepackage{hyperref}

%Information to be included in the title page:
\title{二维正方晶格伊辛模型的蒙特卡洛模拟}
\author{吴晋渊 18307110155}
\institute{复旦大学物理学系}
\date{2021}

\usetheme{Madrid}

\newcommand{\concept}[1]{\textbf{#1}}

\begin{document}

\frame{\titlepage}

\begin{frame}
\frametitle{Finite size scaling}

\begin{itemize}
    \item 一个边长为$L$,正处在其伪临界点$T = T_\text{c}(L)$处的系统等价于一个边长无限长,\emph{关联长度}为$L$,温度为$T_\text{c}$的系统。 % 这个假设是足够强了,可以推导出我们想要的data collapsing公式,问题是它是不是真的是对的
    \item 于是将
    \[
        \xi \propto \abs*{t}^{- \nu}
    \]
    中的$\xi$替换成$L$得到
    \begin{equation}
        L \propto \abs*{T_\text{c}(L) - T_\text{c}(L=\infty)}^{-\nu} .
    \end{equation}
    \item 同样,在
\end{itemize}    

\end{frame}

\end{document}